% !TEX TS-program = XeLaTeX
\documentclass{article}
\usepackage{graphicx}
\graphicspath{ {./images/} }
\usepackage[left=3cm,right=3cm,
    top=2cm,bottom=2cm,bindingoffset=0cm]{geometry}
\usepackage{subfig}
\usepackage{fontspec}
\usepackage{array}
\usepackage{gb4e}
\usepackage{array}
\usepackage{amssymb}

\setmainfont{TeX Gyre Termes}

\title{\textbf{SDRT (Lascarides, Asher): }DRT but make it pragmatic}
\author{Александра Шикунова}
\date{18 мая}

\begin{document}
\maketitle

Введение
Нецензурная лексика всегда была привлекательным объектом для изучения. К этому предмету можно подходить как с внешней стороны, концентрируясь на социальных ситуациях и нормах, с ним связанных, так и с внутренней, языковой стороны.  Обсценная лексика разнообразна; использование некоторых ее элементов строго ограничено до узкого круга неформальных контекстов, другим же позволительно появляться в письменной речи, на телеэкране и т.д.

Но насколько ограничены нецензурные выражения с точки зрения грамматики? Нелитературные (как и “несуществующие”) корни часто не имеют собственного лексического значения и при словообразовании принимают значение аффиксов, которые присоединяют:

укокошить (значение не от корня кокош, а от окружающих аффиксов)
зафигачить (аналогично)

Таким образом, значение того или иного корня может быть любым, а сочетаемость нецензурных лексем - чрезвычайно широкой; неформальный контекст их употребления способствует ее дальнейшему расширению. Тем не менее, синтаксические роли и семантика настолько широко употребляемых выражений не могут быть совершенно произвольны. На корпусном материале будет показано, какие языковые ограничения могут налагаться на обсценные выражения и как они меняются во времени на примере устойчивых конструкций с корнем “фиг” (нифига/ни фига, нефиг/не фиг, фиг). Нифига способно создавать отрицательный контекст, а нефиг и фиг -- модальный.

К счастью, этот корень не настолько нецензурен, чтобы полностью исключаться из печати, поэтому его употребление удобно изучать на материале корпуса письменных текстов. Я буду пользоваться данными из НКРЯ и интернет-узуса.
Содержание
Раздел 2 содержит обзор существующих взглядов на выражения с корнем фиг и словарные толкования. В разделе 3 описаны принципы подбора материала и его источники. В разделе 4 проводится анализ семантики и сочетаемости некоторых выражений с корнем фиг. Итоги подведены в разделе 5.
История вопроса
Навасартова 2017 (ссылка) анализирует словообразовательный потенциал лексемы “фиг” и приходит к выводу, что слова с этим корнем могут иметь очень широкий круг значений:
“
лексема “фига” [...] может придавать различные значения и оттенки в процессе  словообразовании: от непосредственно близких по семантике слов («фигня») до полностью противоположных («офигенный») значений. 
“
Буквальное значение корня фиг не нужно принимать во внимание при исследовании семантики и сочетаемости нифига (и других выражений), ведь даже в толковых словарях для слов фиг и нифига зачастую находим два разных вхождения.

перен. То же, что и кукиш (Ушаков)
Сложенная в кулак рука с большим пальцем, просунутым между указательным и средним — грубый жест в знак презрительного отказа, издевки, насмешки и т. п.; шиш. (МАС)
употр. для выражения отрицания, отказа от чего-л. Купили билеты? — Фиг! Отдай книжку! — Фиг тебе! Запачкал куртку, фиг отчистишь! (Кузнецов, к статье “ни фига”)
Совсем ничего, ни черта (Ожегов)
Сниж. Полное отсутствие чего-л.; ничто, ничего (Кузнецов)
Грубо-прост. Ничего не вышло, не слышно и т. п. (Федоров, фразеологический словарь)

Ни- и -а Навасартова считает префиксом и суффиксом соответственно, хотя выражение нифига до сих пор часто встречается в раздельном написании (ни фига). Таким образом, степень грамматикализации этого выражения остается под вопросом.

В обзоре Навасартовой отсутствуют слова нефиг и фиг (в значении фиг разберешь) и в целом не исследована сочетаемость фиг-выражений.

Фиг подходит под понятие вульгарных минимизаторов Шмелева (ссылка). Он обнаруживает синонимию предложений с отрицанием и без него при использовании таких выражений. Получается, неоднозначно не только значение корня, но и то, какую роль в предложении играют образованные им слова. На примере “минимизатора” фиг будет показано, какие интерпретации он лицензирует.

Нифига/нефиг похожи по форме на отрицательные местоимения по Падучевой (ссылка Падучева 2017).
“
Между тем про отрицательные местоимения есть основания считать, что они сами по себе выражают отрицание. Тогда отрицательное местоимение – это нереферентное неопределенное местоимение в соединении с отрицанием, в сфере действия которого оно находится.
“
Отрицательные местоимения вызывают не-согласование с предикатом (negative concord) и поэтому являются NCI (negative concord items). В этой работе для различения NPI и NCI будут использоваться критерии из Fitzgibbons 2010:

 i. NCIs cannot appear in non-negative contexts, but NPIs can
ii. NCIs but not NPIs can appear in preverbal subject position, not c-commanded by negation
 iii. NCIs but not NPIs can be modified by expressions like almost
iv. NCIs but not NPIs are grammatical in elliptical answers
v. An NCI cannot be licensed across an indicative clause boundary, but an NPI can
Материалы и методы
Была собрана выборка из 297 примеров, взятых из НКРЯ. Примеры относятся главным образом к периоду конца 20-начала 21 века, что означает, что фиг-выражения относительно новы в русском языке. Самое раннее вхождение ни фига в НКРЯ относится к 1925 году (график из НКРЯ?).

Употребления корня фиг в составе фразеологизмов (напр. нифига себе, как нефиг делать, фиг тебе) или междометий (фига се; фиг! в значении отказа) были отфильтрованы, так как такие выражения синтаксически и семантически неразложимы, а значит не помогут нам выявить значение слова с корнем фиг как такового.

В выборке есть 4 класса примеров:
нифига в сочетании с неглагольными предикатами (“нифига тебе не понятно”)
нифига в сочетании с глагольными предикатами (“нифига ты не понял”)
примеры с нефиг (“нефиг тут флудить”)
примеры с фиг (“ее фиг собьешь”)

В своей работе я буду опираться на теоретические инструменты формальной семантики, 
Анализ
В процессе работы с выборкой были обнаружены следующие закономерности в употреблении фиг-конструкций:

Нифига постепенно грамматикализуется и его можно считать словом, а не фразеологизмом.
Нифига в предложении может играть роль наречия либо именной группы; между этими ролями может наблюдаться двусмысленность в сочетании с некоторыми предикатами.
Нефиг обладает некоторыми свойствами негде спать-конструкций (ссылка на Апресяна).
Фиг создает модальный контекст невозможности или маловероятности (то есть содержит в себе отрицание и не является ОПЕ) и сочетается с особой формой глагола, образуя конструкцию.
Нифига: наречие или существительное
Посмотрим на три примера ниже.
Ты оленей видел, нет? Ни фига ты не видел, собачьи упряжки ты видел, а вот оленей тебе не пришлось наблюдать. 

 ― Согласен. Я тоже в эту категорию ни фига не верю. А что есть? 

 ― Блин, целый час ждать, потому что за сорок минут ты не успеешь ни фига. 
В примере 1а глагол видеть переходный, и ни фига занимает тета-роль прямого объекта. В 1б верить не имеет прямого объекта, поэтому ни фига играет роль наречия при глаголе. В 1в доступно два прочтения: если глагол успеть переходный, то ни фига - объект (=не успеешь ничего), а если непереходный - наречие (=совсем/никак не успеешь).

Значение предложения это различие не меняет: остается отрицательный контекст. Тем не менее, в некоторых случаях прочтения с наречием и с существительным существенно различны.

Анджела. Ни фига она не учла. Спрашиваете ― с малахольной. 
Если ни фига - наречие и прямой объект упоминался ранее и опущен в результате эллипсиса, то получается, что героиня не учла что-то одно (сфера действия экзистенциального квантора над отрицанием). Если ни фига - именная группа и прямой объект, то не было учтено ничего (отрицание над квантором). (возможно, стоит показать структуру этого всего)

С неглагольными предикатами разделение тоже присутствует:
 ― Брось, ни фига тебе не понятно. Жена есть?
Но в темноте ни фига ведь не видно. То одна машина обгонит, то другая. 
Ни фига не возможно привыкнуть.

Пример А содержит двусмысленность, знакомую нам из примера выше. В примере Б ни фига - прямой объект, а значит, именная группа, а в примере В - наречие.

Является ли нифига NCI? Проведем тест по критериям Fitzgibbons 2010:

Критерий
Соответствие
Пример с нифига
Аналогичный пример со “стандартным” NCI
 i. NCIs cannot appear in non-negative contexts, but NPIs can
+
*Нифига произошло.
*Ничего произошло.
ii. NCIs but not NPIs can appear in preverbal subject position, not c-commanded by negation
+
Нифига не выйдет.
Ничего не выйдет.
 iii. NCIs but not NPIs can be modified by expressions like almost


+
Она почти нифига не решила.
Она почти ничего не решила.
iv. NCIs but not NPIs are grammatical in elliptical answers
?
(-- Что случилось?)
-- *Нифига.
(-- Что случилось?)
-- Ничего.




(-- Сколько задач она решила?)
-- ?Нифига.
(-- Сколько задач она решила?)
-- Нисколько.
v. An NCI cannot be licensed across an indicative clause boundary, but an NPI can
+
Я не думала, что нифига *(не) выйдет.
Я не думала, что ничего *(не) выйдет.

Нифига проходит все тесты на NCI, кроме эллиптического. Это может быть связано с тем, что у нифига не фиксирована синтаксическая роль. Как мы видели ранее в примере N, оно допускает двусмысленность. Помимо этого, сложно сказать, на какой вопрос более приемлемо ответить “Нифига.”, будет оно именной группой или же обозначать степень или количество, как нисколько.

Итак, нифига постепенно грамматикализуется, присоединяясь к классу слов русского языка, которые Fitzgibbons 2010 называет n-words: никто, ничего, никак, нигде и т.д. Этот процесс можно проследить по написанию этого выражения: чем новее вхождение в корпусе, тем более вероятно, что оно написано слитно. (взять графики из НКРЯ)
Нефиг спать
Слово нефиг ново (первое вхождение в НКРЯ - 1986 год) и интересно с точки зрения его сочетаемости. Во-первых, оно образует распространенный фразеологизм “как нефиг делать”, который при сборе примеров был отброшен, но к нему нам придется вернуться для объяснения природы самого нефиг. Во-вторых, внешне и просодически это слово напоминает негде спать-конструкции из Апресян ГГГГ (ссылка).

Попробуем разделить нефиг на 5 компонентов, предложенных Апресяном.

Позиция
Наполнение
Пример с нефиг
Соответствие
Именная группа, агенс
имя (местоимение/существительное) в дативе
Нефиг тебе спать до трех часов.
+
Глагол
экзистенциальный/посессивный глагол (не совсем свободно)
см. ему не нашлось что сказать
*Мне не было фиг делать.
-
Актант инфинитива
К-слово, соответствующее семантике инфинитива
Мне нефиг делать.(нефиг=нечего)
?Мне нефиг поесть (в значении “мне нечего есть”)
?
Глагол
быть1 (глагол-связка) и связочные глаголы, если во 2 позиции есть не2; иначе - нулевая форма быть1
нефиг тебе было спать до трех часов
+
Глагол
инфинитив, обозначающий действие, агенс - первый актант глагола
нефиг тебе было спать до трех часов
+

Соответствие негде спать-конструкции полноценно до тех пор, пока дело не доходит до наиболее ограниченных компонентов, а именно экзистенциального глагола и К-слова. 

Первый способен “разорвать” не-слово (см. пример), отделив К-слово. Нефиг же фразеологизировано настолько, что фиг не может быть самостоятельным К-словом, и поэтому замена нулевого экзистенциального глагола на другой не допускается. Что касается актанта инфинитива (К-слова), как мы уже видели на примере нифига, семантика этой составляющей нефиг недоопределена, поэтому можно предположить по аналогии, что это и есть корень проблемы. Но, как мы увидим далее, в значении как именной группы, так и наречия, сочетаемость нефиг весьма ограничена. Именно высокий уровень идиоматизации нефиг отделяет его от класса негде спать-конструкций -- оно неразрывно.

Где нефиг может играть роль именной группы? Вернемся к фразеологизму “(как) нефиг делать”. Именно в сочетании с глаголом делать (и, насколько позволяет считать выборка, ни с каким другим) нефиг может играть роль прямого объекта. Позицию субъекта оно не занимает никогда, как и любые другие негде спать-конструкции (она обязательно занята именной группой в дативе). Поскольку нефиг не изменяется по падежам, оно не может играть роль и непрямого объекта или разнообразных адъюнктов. Таким образом, единственный контекст, в котором нефиг может быть именной группой -- идиоматизированное сочетание “нефиг делать”.

Что касается наречной функции, ее нефиг тоже выполняет по-особому. Рассмотрим несколько примеров.
 ― Для нее и Катя сойдет. И вообще, не фиг всяким бабкам лезть к моей девушке. 

Нефиг себя из-за мужиков уродовать.

Нефиг всегда выдвигается на левую периферию предложения, что говорит о его топикализации. Элицитация показывает, что в предложениях, где нефиг выражает деонтическую модальность, оно не может занимать другую позицию. 

Нефиг ему спать до обеда.
*Ему нефиг спать до обеда.

Также нефиг сочетается исключительно с глаголами несовершенного вида. Это характерно для предложений с отрицанием. 

“
при отрицании совершенный вид финитного глагола может меняться на несовершенный – такая замена семантически мотивирована, но, как правило, факультативна. Есть, однако контекст, где замена СВ на НСВ при отрицании обязательна. Это контекст отрицаемой модальности, см. Рассудова 1982: 120-127.
Отрицание в контексте модальности необходимости, общей и деонтической, требует инфинитива глагола в несов. виде
“
(Падучева про отрицание 6.2.3)

В случае с нефиг у всех контекстах его употребления в качестве наречия действительно присутствует модальность необходимости: их объединяет значение “не надо/не стоит N делать V”.

Подводя итоги, нефиг сложно отнести к негде спать-конструкциям из-за его высокой идиоматизации и, как следствие, ограниченных контекстах употребления. Нефиг-наречие создает модальный контекст необходимости, и это накладывает ограничение на вид инфинитива.

Новизна выражения может быть причиной узкой сочетаемости. До 2003 года в корпусе встречается написание нефига, которое может указывать на его происхождение от нифига. К сожалению, данных (в том числе просодических) недостаточно, чтобы точно это установить. Зато тенденция к слитному написанию присутствует, как и у нифига.
Фиг разберешь
Общие наблюдения
Фиг в составе процитированной в названии раздела конструкции, как и нефиг, создает модальный контекст, но совсем другой. Рассмотрим примеры:

Они на нее орут, но ее фиг собьешь! 
 ― Но спрячут так далеко, что фиг найдешь. 

Предложения с фиг объединяет значение “V маловероятно/невозможно”. Иногда в них появляется субъект в номинативе или меняется лицо/время глагола:

Я вот и выпила супрастина, но к врачу фиг попаду в ближайшие дни, потому что дитенок мой болеет, я с ним сижу ((
Сейчас зашлют хрен знает куда, получка фиг дойдет. 
Все-то ему можно ― меня бы еще фиг выпустила. 

В основном же ограничения на конструкцию таковы: субъект опционален, но обязательно в номинативе; предикат глагольный, глагол совершенного вида, чаще всего в будущем времени, 2 лица, единственного числа (но возможна вариация). Отклонения лицензируются субъектом пропозиции, а если субъект опущен, глагол получает форму “по умолчанию”. (отсылка к кросс-лингвистической закономерности?)

Фиг представляется единственным из трех выражений, анализируемых в этой работе, которое не просто не расширяет свою сочетаемость, а сужает её: примеры с отклонениями в лице и числе сказуемого редки и относятся, самое позднее, к 2006 году.

Фиг и информационная структура
Обратим внимание на положение фиг в следующих предложениях:

Фиг он придет.
?Он фиг придет.
?Фиг её найдешь.
Ее фиг найдешь.
Фиг ты ее найдешь.

Фиг находится на левой периферии предложения, за исключением случаев, когда топикализуется объект или другие составляющие. Элицитация показывает, что фиг предпочтителен в начале предложения, а если перед ним стоит субъект, и так традиционно топикальный (ссылка про информационную структуру?), предложение становится неграмматичным.

В примерах из выборки встречаются разные топикализованные составляющие:
(вставить примеры того, что может топикализоваться)

Интересно, что фиг сочетается с к-словами:
― От самих от них фиг чего дождешься.
А если ты мужик, то тебя фиг кто грабанет.

Это явление показывает, что фиг действительно создает неверидикативный контекст, в котором возможно появление к-слов как NPI.
Тип модальности 
Проведем тест фиг на сочинение.

Преобразование
Приемлемость
Пример с другим модальным оператором
Пример с фиг
Сочинение
+
Её невозможно найти и/или поймать.
Нельзя, чтобы ты пошла, а она осталась.
ОК Её фиг найдёшь и/или поймаешь.
*Фиг ты пойдёшь, а она останется.
?{Я иду гулять, а мама настаивает, чтобы я взяла с собой младшую сестру} Фиг ты пойдешь, если она останется.
Сочинение при фиг (пример Х) носители считают неграмматичным, в отличие от подчинения (пример У). Это значит, что в сфере действия фиг может быть не более одной клаузы. При этом сочинение предикатов вполне допустимо.

Заметим также, что фиг сочетается только с формой будущего времени (в выборке есть 1 вхождение в НКРЯ с сослагательным наклонением; носители не считают фиг-предложения с прошедшим временем грамматичными).

Туда фиг залезешь.
Фиг она туда залезет.
?Фиг она туда будет/станет залазить.
*Фиг она туда залезла.

Чем можно объяснить такие ограничения? Дело в типе модальности. Von Fintel \& Heim 2021 предполагают, что существует два типа модальности: эпистемическая и корневая. Часто оба типа выражаются одним и тем же модальным оператором, и устранить двусмысленность может только контекст. Я сделаю попытку на нескольких примерах показать, какой modal flavor может фигурировать в предложениях с фиг.

Они на нее орут, но ее фиг собьешь!
Утопишься в Неве ― фиг найдут.
А корреспонденту они фиг посмеют отказать… Я так думаю.

Эпистемическая модальность принимает в качестве modal flavor знания говорящего о мире. Так как фиг-предложения описывают ситуацию в будущем, а знаний о будущем у нас нет, модальность фиг не эпистемическая, а динамическая (dynamic/circumstantial по Von Fintel \& Heim 2021). Динамическая модальность обозначает, что пропозиция оценивается относительно законов природы, того, как бывает в мире, и какие предсказания мы можем сделать на основании этого.

Так как modal flavor фиг в случае корневого прочтения берется из субъекта внутренней клаузы, если таких клауз несколько, один modal flavor не подойдет под все, и предложение окажется неграмматичным.

Единственный пример с сослагательным наклонением в сочетании с фиг, приведенный ниже, тоже можно проанализировать через динамическую модальность.

Все-то ему можно ― меня бы еще фиг выпустила.

В русском языке сослагательное наклонение означает, что действие не реализовано на некоторый момент времени, берущийся из контекста (Croft 1975). Такое значение совместимо с семантикой фиг: эпистемическое прочтение всё еще исключено, так как сослагательное наклонение не выражает знания о мире, как и будущее время.

В пользу динамического прочтения фиг-предложений говорит и феномен, который замечают Von Fintel \& Heim 2021 (проверить элицитацией про вид)

Итак, фиг выражает динамическую модальность, этим объясняется его сочетаемость. Фиг встречается с глаголами в будущем времени или в сослагательном наклонении, может принимать пропозиции с различными субъектами, хотя наиболее распространенный вариант -- опущенный 2sg.
Summary
Было рассмотрено 3 выражения с корнем фиг: нифига, нефиг и фиг. Все они встречаются в отрицательных контекстах, но у каждого особое значение и разные синтаксические роли. Нифига является negative concord item и близко к русским н-словам, таким как никто, никогда, никак и пр. Предложения с нефиг содержат негативную деонтическую модальность; связь нефиг с негде спать-конструкциями опосредованна. Фиг создаёт оттенок негативной динамической модальности.

Выражения нифига и нефиг грамматикализуются, судя по тенденции к слитному написанию и по утере корнем фиг лексического значения. Такой путь грамматикализации отрицательных частиц не нов -- известен пример французского pas, которое уже давно не означает “шаг”, а является частью отрицания.

В силу незначимой лексической составляющей выражения с фиг ограничены главным образом синтаксически. Именно поэтому конструкции с этими выражениями расширяют свою сочетаемость неохотно, но в то же время различные нецензурные корни (фиг, хрен, шиш и еще менее печатные варианты) заменяются друг на друга практически свободно. Обобщение, которое можно сделать на основании моей работы, таково: ограничения, накладываемые на сочетаемость нецензурных выражений связаны не со значением корней самих по себе, но с синтаксическими и семантическими ролями, которые они могут играть. Обсценные корни по отдельности же действительно обладают чрезвычайно широкой сочетаемостью и практически неопределимым лексическим значением. Поэтому изучение нецензурной лексики стоит свести к изучению структуры предложений, в которые она входит, а не к поиску собственных значений нелитературных лексем.

\end{document}

